\documentclass[11pt,a4paper]{article}

\usepackage[slovene]{babel}
\usepackage[utf8x]{inputenc}
\usepackage{graphicx}
\usepackage{hyperref}
\usepackage{pdfpages}
\pagestyle{plain}


\begin{document}
\title{Poročilo pri predmetu \\
Analiza podatkov s programom R}
\author{Dževada Džafić}
\maketitle

\section{Brezposelnost v Sloveniji}

Tema mojega projekta je brezposelnost v Sloveniji. Osredotočila sem se na analizo brezposelnosti po naslednjih spremenljivkah: trajanje iskanja dela, spol in starost po statističnih regijah Slovenije v letu 2014.

Povezava do podatkovnih tabel:
\begin{itemize}

\item \url{http://pxweb.stat.si/pxweb/Dialog/varval.asp?ma=0762111S&ti=&path=../Database/Dem_soc/07_trg_dela/02_07008_akt_preb_po_anketi/02_07621_akt_preb_ADS_letno/&lang=2}
\item \url{http://www.ess.gov.si/trg_dela/trg_dela_v_stevilkah/registrirana_brezposelnost#Statisti%C4%8Dne%20regije}

\end{itemize}

Cilj projekta je predvsem spoznati delo v programskem okolju R ter analizirati brezposelnost v Sloveniji glede na trajanje iskanja dela, spol in starost.


\section{Obdelava, uvoz in čiščenje podatkov}

V drugi fazi projekta sem uvozila 3 tabele v obliki CSV ter narisala 2 grafa.
Prva tabela prikazuje brezposlenost po statisičnih regijah Slovenije glede na trajanje brezposlenosti.
V drugi tabeli sem prikazala brezposlenost po statističnih regijah Slovenije glede na spol. Tretja tabela pa prikazuje brezposelnost po statističnih regijah Slovenije glede na starost brezposelnih.
V prvem grafu sem prikazala trajanje brezposlenosti po statističnih regijah v Sloveniji v letu 2014. V drugem grafu pa je prikazana brezposlenost prebivalcev starejših od 55 let po statističnih regijah v letu 2014.

\makebox[\textwidth][c]{
\includegraphics[width=2.5\textwidth]{../slike/grafi1.pdf}
}

\makebox[\textwidth][c]{
\includegraphics[width=2.5\textwidth]{../slike/grafi2.pdf}
}
\newpage

\section{Analiza in vizualizacija podatkov}

V tretji fazi projekta sem narisala zemljevid Slovenije, ki prikazuje brezposelnost v Sloveniji po statističnih regijah v letu 2014. 

\makebox[\textwidth][c]{
\includegraphics[width=1.7\textwidth]{../slike/Slovenija.pdf}
}

% 
% \section{Napredna analiza podatkov}
% 
% \includegraphics{../slike/naselja.pdf}

\end{document}

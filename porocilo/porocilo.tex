\documentclass[11pt,a4paper]{article}

\usepackage[slovene]{babel}
\usepackage[utf8x]{inputenc}
\usepackage{graphicx}

\pagestyle{plain}

\begin{document}
\title{Poročilo pri predmetu \\
Analiza podatkov s programom R}
\author{Dževada Džafić}
\maketitle

\section{Brezposelnost v Sloveniji}

Tema mojega projekta je brezposelnost v Sloveniji. Najprej se bom osredotočila na analizo brezposelnosti po naslednjih spremenljivkah: trajanje iskanja dela, spol, kohezijska regija in leto (1. tabela). Potem bom še analizirala brezposelnost v Sloveniji po izobrazbi in regijah v letu 2014 (2. tabela).

Povezava do podatkovnih tabel:
http://pxweb.stat.si/pxweb/Dialog/varval.asp?ma=0762111S&ti=&path=../Database/Dem_soc/07_trg_dela/02_07008_akt_preb_po_anketi/02_07621_akt_preb_ADS_letno/&lang=2
http://www.ess.gov.si/trg_dela/trg_dela_v_stevilkah/registrirana_brezposelnost#Statisti%C4%8Dne%20regije

Cilj projekta je predvsem spoznati delo v programskem okolju R ter analizirati brezposelnost v Sloveniji glede na trajanje iskanja dela in izobrazbo.


\section{Obdelava, uvoz in čiščenje podatkov}

\section{Analiza in vizualizacija podatkov}

\includegraphics{../slike/povprecna_druzina.pdf}

\section{Napredna analiza podatkov}

\includegraphics{../slike/naselja.pdf}

\end{document}
